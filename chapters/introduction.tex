%---------------------------------------------------------

El Reconocimiento de la Actividad Humana\cite{Humanactivityrecognition} (Human Activity Recognition) es un tema que se ha sido objeto de distintos estudios e investigaciones. 
Human Activity Recognition o como se conoce en sus siglas, HAR, se define como la habilidad de interpretar los movimientos o gestos humanos a través de sensores para reconocer la actividad o acción humana.\cite{10.1007/s11390-011-9430-9}

Este campo de investigación es clave dentro del campo de la computación ubicua\cite{10.5555/2832747.2832806}, y a partir de estos estudios se pueden desarrollar herramientas para el desarrollo de las personas en áreas como sistemas de vigilancia\cite{10.1007/s11390-011-9430-9}, la salud\cite{10.1145/3195106.3195157}, interacción humana con computadores\cite{10.1007/s11390-011-9430-9}, y cualquier campo relacionado donde sea aplicable. \newline
Uno de los campos que abre este tipo de investigaciones el de métodos de transporte\cite{Efthymiou2019}. Los métodos de transporte han abierto un campo dentro de la computación ubicua. Debido a la aparición de nuevos dispositivos móviles como smartphones y weareables, se han abierto nuevas oportunidades para la exploración de este tema.



\section{Brecha de Conocimiento (Qué falta por hacer)}

La brecha de conocimiento existente tiene relación con el uso de smartwatch para realizar las mediciones. Generalmente las investigaciones que se han realizado se han hecho con smartphones. A pesar de que un smartphone y un smartwatch posee sensores similares, los resultados podrían variar. Se tiene como referencia papers anteriores donde muestran que los resultados varían respecto donde se utiliza el sensor para detectar la actividad humana.
Por otro lado, las investigaciones realizadas se han hecho con métodos de transporte usuales que no tienen mucha relación con actividades deportivas, por lo que se abre una brecha de conocimiento en este apartado. Generalmente se utilizan métodos de transporte usuales como bus, auto, bicicleta, correr, caminar. No se ha indagado en investigaciones con otros métodos de transporte por ejemplo kayak, scooter, skate, paracaidismo, etc. 


\section{Decidir bien si esta relacionado al tipo de dato o tipo de actividad (Actividad Humana de Esparcimiento}

En las investigaciones realizadas se han utilizado distintos tipos de datos, algunos de estos con sensores GPS, otros con acelerómetros, giroscopios, magnetómetros, altímetros, entre otros. En la mayoría de los casos, se utilizan grandes escalas de datos que han sido recolectadas por usuarios de prueba

\section{Preguntas de Investigación}
\begin{enumerate}
	\item ¿Qué datos se deben utilizar?
	\item ¿Cuántas actividades abarcará la investigación?
	\item ¿Qué técnicas de Machine Learning se utilizarán para resolver el problema?
	\item ¿Cuál será la frecuencia de datos que escogerá?
	\item Bajo que dispositivos se obtienen mejores resultados \textcolor{red}{(Se podría hacer una comparación con las investigaciones anteriores)}
\end{enumerate}

\section{Propósito y Objetivos}
\subsection{Propósito}
El propósito de este trabajo es ofrecer una solución de reconocimiento de Métodos de Transporte utilizando dispositivos de consumo masivo y de fácil acceso. En este caso, se obtendrá el flujo de datos a través de un Smartwatch, debido al uso masivo que abarca este dispositivo.

\subsection{Objetivos}
El objetivo general de esta investigación es ofrecer 
\begin{enumerate}
	\item Comprender cómo los métodos de transporte se relacionan con las distintas actividades humanas
	\item Usar hardware disponible de consumo masivo, en este caso un Smartwatch para la obtención de datos.
	\item Utilizar datos reales para el entrenamiento del algoritmo de Machine Learning.
	\item 
\end{enumerate} 

\section{Metodología}




