%---------------------------------------------------------
% \section{Background}

El Reconocimiento de la Actividad Humana\cite{Humanactivityrecognition} (Human Activity Recognition) es un tema que se ha sido objeto de distintos estudios e investigaciones. 
Human Activity Recognition o como se conoce en sus siglas, HAR, se define como la habilidad de interpretar los movimientos o gestos humanos a través de sensores para reconocer la actividad o acción humana.\cite{10.1007/s11390-011-9430-9}

Este campo de investigación es clave dentro del campo de la computación ubicua\cite{10.5555/2832747.2832806}, y a partir de estos estudios se pueden desarrollar herramientas para el desarrollo de las personas en áreas como sistemas de vigilancia\cite{10.1007/s11390-011-9430-9}, la salud\cite{10.1145/3195106.3195157}, interacción humana con computadores\cite{10.1007/s11390-011-9430-9}, y cualquier campo relacionado donde sea aplicable. \newline
Uno de los campos que abre este tipo de investigaciones el de métodos de transporte\cite{Efthymiou2019}. Los métodos de transporte han abierto un campo dentro de la computación ubicua. Debido a la aparición de nuevos dispositivos móviles como smartphones y weareables, se han abierto nuevas oportunidades para la exploración de este tema.



\section{Brecha de Conocimiento (Qué falta por hacer)}

La brecha de conocimiento existente tiene relación con el uso de smartwatch para realizar las mediciones. Generalmente las investigaciones que se han realizado se han hecho con smartphones. A pesar de que un smartphone y un smartwatch posee sensores similares, los resultados podrían variar. Se tiene como referencia papers anteriores donde muestran que los resultados varían respecto donde se utiliza el sensor para detectar la actividad humana.
Por otro lado, las investigaciones realizadas se han hecho con métodos de transporte usuales que no tienen mucha relación con actividades deportivas, por lo que se abre una brecha de conocimiento en este apartado. Generalmente se utilizan métodos de transporte usuales como bus, auto, bicicleta, correr, caminar. No se ha indagado en investigaciones con otros métodos de transporte por ejemplo kayak, scooter, skate, paracaidismo, etc. 





%---------------------------------------------------------

%\section{Problem statement}


%\section{Objectives}

%\subsection{General Objective}



%\subsection{Specific Objectives}


%\section{Justification}
