%---------------------------------------------------------
% \section{Background}

El Reconocimiento de la Actividad Humana\cite{Humanactivityrecognition} (Human Activity Recognition) es un tema que se ha sido objeto de distintos estudios e investigaciones. 
Human Activity Recognition o como se conoce en sus siglas, HAR, se define como la habilidad de interpretar los movimientos o gestos humanos a través de sensores para reconocer la actividad o acción humana.\cite{10.1007/s11390-011-9430-9}

Este campo de investigación es clave dentro del campo de la computación ubicua\cite{10.5555/2832747.2832806}, y a partir de estos estudios se pueden desarrollar herramientas para el desarrollo de las personas en áreas como sistemas de vigilancia\cite{10.1007/s11390-011-9430-9}, la salud\cite{10.1145/3195106.3195157}, interacción humana con computadores\cite{10.1007/s11390-011-9430-9}, y cualquier campo relacionado donde sea aplicable. \newline
Uno de los campos que abre este tipo de investigaciones el de métodos de transporte\cite{Efthymiou2019}. Los métodos de transporte han abierto un campo dentro de la computación ubicua. Debido a la aparición de nuevos dispositivos móviles como smartphones y weareables, se han abierto nuevas oportunidades para la exploración de este tema.





%---------------------------------------------------------

%\section{Problem statement}


%\section{Objectives}

%\subsection{General Objective}



%\subsection{Specific Objectives}


%\section{Justification}
